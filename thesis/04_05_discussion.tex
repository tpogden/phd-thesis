\section{Discussion}
  \label{sec:polaritons_discussion}

  The \textsc{eit} technique provides an alternative method to \textsc{sit} for
  transmission of light in an optically dense medium, with more control over its
  associated effects such as slow light and pulse compression. We considered the
  transparecnty effect by looking at the time-evolution of the atomic density
  matrix elements and the steaydy state lineshape in the weak probe
  approximation.

  By transforming to the \textsc{cpt} basis we were able to undesrstand
  \textsc{eit} as a coherent effect based upon popuation of the dark-state
  superposition. Introducing the polariton quasiparticle allows us to understand
  this propagation and presents the possibility of storing and retrieving light
  pulses.

  The ability to `stop' light is clearly interesting from a purely scientific
  perspective, but the fact that information encoded in the pulse can be
  reversibly transferred to long-lived spin waves as important applications.
  Significantly, it may be shown that the dark-state polariton picutre also
  holds for quantised light fields, such that indiviidual photon wave packets
  can be stored and retrieveed.[ref lukin] This provides a mecnahism for quanutm
  memories, a key requirement for quantum information processing[ref]. The high
  fidently of the \textsc{eit} memory scheme compares favourably with other
  proposals such as cavity \textsc{qed} and photon echo techniques[ref].

  Finally, by comping a probe transition to Rydberg states in a $\Xi$-type
  sysstem, polaritons can be made to interact due to strong dipole-dipole
  interactions bwteeen such highly excited states.
