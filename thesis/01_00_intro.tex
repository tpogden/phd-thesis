\chapter[Introduction]
  {Introduction}
  \label{chp:intro}

  \begin{displayquote}
    \textit{
    We all \textrm{know} what light is; but it is not easy to \textrm{tell} what
    it is.
    }
    \begin{flushright}
      --- Samuel Johnson.
    \end{flushright}
  \end{displayquote} 

  \begin{displayquote}
  \textit{
  I have also a paper afloat with an electromagnetic theory of light which, til
  I am convinced to the contrary, I hold to be great guns.
  }\\
    \begin{flushright}
      --- James Clerk Maxwell, 5 January 1865.
    \end{flushright}
  \end{displayquote}

  % \subsection*{Intro}

    The propagation of light through resonant matter underlies a huge array of
    physical phenomena. Naturally occurring atmospheric effects such as Rayleigh
    scattering, rainbows, auroras and sun dogs are well known and pleasant
    diversions.

    % The question of how light travels and interacts with matter is fundamental
    % to our understanding of the world, and it has thus been considered by the
    % most prominent natural philosophers including Galileo, al-Haytham, Newton,
    % Maxwell, Dirac and Feynman. Despite this long history, atom-light
    % interactions remains a focus of research, as we can see by the award of the
    % 2012 Nobel prize in physics to Haroche and Wineland for their work with
    % atoms and ions in optical cavities.\cite{Nobelprize.org}

    The question of how light travels and interacts with matter is fundamental
    to our understanding of the world, and it has thus been considered since
    antiquity. Despite this long history, atom-light interactions remains a
    focus of research, as we can see by the award of the 2012 Nobel prize in
    physics to Haroche and Wineland for their work with atoms and ions in
    optical cavities.\cite{Nobelprize.org}

    Modern research has been greatly facilitated by continued development of
    laser technology, which provides sources of intense, coherent and
    monochromatic light for experimental study. This has allowed new and
    interesting nonlinear optical phenomena to be observed in the laboratory,
    providing both a greater understanding of atom-light interaction and novel
    applications in communications\cite{Kimble2008} and quantum information
    processing\cite{Lvovsky2009}.

    The regime of nonlinear optics is defined to be that in which the optical
    response of the medium is modified by the presence of the light itself, such
    that the response depends on the intensity of the light in a nonlinear
    manner.\cite{boyd2008nonlinear} In this thesis we will investigate
    theoretically the propagation of coherent pulses of light in nonlinear
    media.

    % , and in particular two dramatic effects: optical solitons that
    % propagate without distortion, and electromagnetically induced transparency,
    % in which a second, coupling field renders a medium transparent to the pulse.

  % \subsection*{Propagation of Light in Atomic Vapours}

    In order to make a complete description of the interaction of the atoms that
    constitute a vapour with incident light, we should treat both as quantal
    entities. The atoms should be described as charges moving in quantised
    energy levels, and the light as a set of travelling wave modes each with
    associated quantum harmonic oscillators whose excitation number expresses
    the number of photons it contains.\cite{scully1997quantum} The appropriate
    formalism for this complete description is quantum electrodynamics
    (\textsc{qed}), in which the interaction is described by exchange of quanta
    between the atoms and light.

    However, as we are interested here in high-intensity optical fields
    containing a large number of photons, the light behaves in a way that is
    sufficiently non-quantum for us to treat it as a classical electromagnetic
    field.\cite{jackson1998classical} We maintain the quantum description of the
    atoms, but do not treat each atom individually, considering the collection
    of atoms at each point along the propagation axis as a statistical ensemble,
    with parameters such as velocity and energy state given as distributions.

    Where we cannot neglect the quantum nature of light is in considering the
    important effect of spontaneous emission, whereby photons are non-
    deterministically emitted from an atom into the vacuum field modes. We
    include this by treating the decay process statistically, averaging over
    individual atomic emissions to the environment.\cite{loudon2000quantum}

    We couple the Maxwell wave equation describing propagation of the classical
    field to the Lindblad master equation describing the open quantal atomic
    system. To account for thermal motion of atoms, we average their response
    over a velocity distribution.\cite{foot2005atomic}

    In this thesis we will consider in particular two important nonlinear
    effects: self-induced transparency and electromagnetically induced
    transparency. We will go on to demonstrate that in combination these effects
    provide a potentially useful means of propagating pulses in dense thermal
    atomic vapours.

% \subsection*{Nonlinear Susceptibility}

  \subsection*{Self-induced Transparency}

    Self-induced transparency (\textsc{sit}) is a nonlinear phenomena in which
    short, strong pulses with a specific (sech-type) profile and pulse area
    ($2\pi$) are able to travel through an absorptive medium without
    distortion.

    As this optical soliton moves through the medium, its shaped leading edge
    is absorbed and inverts the atomic population, but its trailing edge then
    rotates the population back to the ground state via stimulated
    emission.\cite{allen1975optical} If this process happens on a timescale that
    is much shorter than the decay lifetime of the atoms, the pulse retains
    phase memory and energy is conserved as it propagates. The group velocity is
    delayed by the non-zero amount of time the pulse spends as an excitation as
    it travels.

    The first paper describing the effect was by McCall and Hahn in
    1969.\cite{McCall1969} As well as an analytic description, results were
    presented from an experiment using a liquid-helium cooled ruby absorber
    showing that intense light can be transmitted without attenuation but
    delayed.

    The pulse is actually robust and will form the sech-type profile even if it
    is initially of a different profile, as long as the pulse area is large
    enough. The area theorem\cite{McCall1969} tells us that pulses with area $>
    3\pi$ will break up into multiple solitons that travel at distinct group
    velocities.

    Gibbs and Slusher presented further results from experiments in rubidium
    vapours,\cite{Slusher1972} including both large pulse delays and pulse
    breakup in agreement with the theoretical prediction.

  \subsection*{Electromagnetically Induced Transparency}

    Electromagnetically induced transparency (\textsc{eit}) is a technique for
    allowing a probe light pulse to propagate through a medium it would
    ordinarily find to be opaque, facilitated by a second, coupling pulse on a
    connected transition forming a three-level system.\cite{Fleischhauer2005}

    It is a coherent effect that can be understood as interference between
    excitation channels. From another perspective, we can transform to a
    dressed state basis and understand \textsc{eit} as originating in the
    formation of a population of a dark-state superposition, decoupled from
    excitation. This results in a narrow transmission window in the probe
    absorption lineshape, which is nonlinear.

    The introduction of \textsc{eit} as a nonlinear optical process was made by
    Harris \etal in 1990\cite{Harris1990}, though Harris notes\cite{Harris1997}
    that the `essence of \textsc{eit}' is in coherent population trapping,
    discovered in 1976 by Alzetta \etal\cite{Alzetta1976}. Experimental
    observation of \textsc{eit} in a strontium vapour was made by Boller
    \etal\cite{Boller1991} in 1991.

    At the same time as absorption is reduced, the dispersive properties of the
    medium are inverted, significantly reducing group velocity on resonance.
    Pulses are then slowed in the medium. In experiments with Bose-Einstein
    condensates (\textsc{bec}s), group velocities have been reduced to 
    $\unit[17]{m/s}$ by Hau \etal\cite{Hau1999}.

    With a time-dependent coupling field, Fleishhauer and Lukin showed that it
    is even possible to store and retrieve a pulse in a medium using
    \textsc{eit}, and introduced the concept of the dark state
    polariton\cite{Fleischhauer2000}. This provides a possible implementation
    for quantum memory using photonic qubits\cite{Lvovsky2009}.

  \section{Thesis Structure}
  \label{sec:intro_structure}

  The remaining chapters of this thesis are structured as follows:

  \begin{description}
  \item[Chapter \ref{chp:propagation}]

    We derive a semiclassical model for propagation of light in thermal atomic
    vapours based on the Maxwell-Bloch equations. We introduce linear and
    nonlinear susceptibilities and discuss analytic results available under the
    weak probe approximation.

  \item[Chapter \ref{chp:nonlinear}]

    We take the model into the regime of nonlinear optics, demonstrating some
    effects that emerge from the interaction of strong fields with atomic
    vapours, notably self-induced transparency and simultons.

  \item[Chapter \ref{chp:polaritons}]

    We introduce the well-known phenomenon of electromagnetically induced
    transparency and the related quasiparticle known as the dark-state
    polariton. We describe how such systems may be used to store and retrieve
    light pulses.

  \item[Chapter \ref{chp:twophoton}]

    We investigate the interaction of a high-intensity beam with a thermal
    vapour of rubidium to model experimental results showing population of
    highly-excited 5d states. We include angular momentum structure and
    broadening effects and consider two-photon excitation as a possible
    mechanism.

  \item[Chapter \ref{chp:simultons}]

    We present the key results of this thesis. We describe a scheme to combine
    the nonlinear phenomena of optical solitons and \textsc{eit} to propagate
    robust simultaneous pulses (simultons) in V-type media, showing excellent
    agreement with experimental results over a range of powers and temperatures.
    This scheme avoids the requirement of high-intensity pulses in the
    \textsc{sit} system, and shows that weak field soliton components may even
    be drawn from a continuous wave field.

  \item[Chapter \ref{chp:conclusions}]

    We conclude, summarise the results and suggest future directions for
    continuing the research presented.

\end{description}

  \section{Publications Arising from This Work}
  \label{sec:intro_pubs}

  T. P. Ogden \etal, \textit{Formation of Simultons in an Atomic Vapor}. [In preparation]