\section{Simulating Longer Propagation Distances}
  \label{sec:simultons_long}

    Thus far we have considered the behaviour of the atomic medium as addressed
    by the \textsc{cw} probe and disturbed by the strong pulse over the
    propagation distance of the thin cell. The restricted propagation distance
    is a limit of the current experimental setup, but in our numerical
    simulations we are not subject to the same constraint. We may extend the
    propagation medium arbitrarily far to observe what happens to both the atoms
    and the propagating fields. This will complete our analysis of the observed
    signal response, and also allow us to make predictions for future laboratory
    studies.

  \subsection{Propagation in the Coupling Pulse Scheme}

    We will first consider some demonstration cases, before looking at the
    specific parameters for the experimental system. We know from the study of
    two- and three-level media in chapter \ref{chp:nonlinear} that a key
    property in the propagation of short pulses in nonlinear systems is the
    pulse area $\theta$ defined in equation \ref{eqn:pulse_area}. Thus we'll
    design simulations with fixed input pulse areas, rather than specifying the
    peak intensities as we have done so far. Of course, for a given Gaussian
    pulse width, these definitions are interchangeable.

    We will for now neglect the motional and hyperfine pumping effects we added
    to the model in section \ref{sec:simultons_theory}, as we seek to gain
    physical insight into the specific effects of propagation.

    \begin{figure}[h]
    \includegraphics[width=\linewidth]{figs/06_simultons/mb_vee_sit_plot_08pi_18pi_fig1.pdf}
    \caption{
    Propagation of (a) $0.8 \pi$ and (b) $1.8 \pi$ coupling pulses (blue)
    through a medium addressed by a $\unit[10]{\Gamma}$ \textsc{cw} probe
    (green), showing (top) profiles of the real part of the complex Rabi
    frequencies $\Omega(z, \tau)$ and (bottom) pulse areas $\theta(z)$.
    }
    \label{fig:pulse_cw_08pi_18pi}
    \end{figure}

    In figure \ref{fig:pulse_cw_08pi_18pi} we present numerical results for the
    \textsc{cw} probe, coupling pulse scheme in a medium with absorption
    coefficients set at $N g_{01} = N g_{02} = \unit[2\pi~10^3]{\Gamma/L}$. This
    is an order of magnitude larger than those representing the thin cell
    experiments, and therefore represents a longer distance of propagation. The
    coupling input pulses are Gaussians of width $\tau_w = $
    \unit[0.01]{$\tau_\Gamma$} and have pulse areas of (a) $0.8 \pi$
    (corresponding to a peak $\Omega_c = \unit[2\pi~28]{\Gamma}$) and (b) $1.8
    \pi$ (a peak $\Omega_c = \unit[2\pi~62]{\Gamma}$). In both cases the
    \textsc{cw} probe is strong with Rabi frequency $\Omega_p =
    \unit[2\pi~10]{\Gamma}$.

    In figure \ref{fig:pulse_cw_08pi_18pi}(a) we see that for the $0.8 \pi$ pulse both
    the \textsc{cw} probe and the coupling pulse are absorbed close to the front
    of the medium, with the pulse area dissipated by around $z = $
    \unit[$0.1$]{$L$}. From then on the only remnant of the fields is the fast
    ringing.

    In figure \ref{fig:pulse_cw_08pi_18pi}(b) we see that for the $1.8 \pi$
    pulse, the large coupling pulse kicks up a pulse from the \textsc{cw} field,
    consistent with our analysis of a period of reduced absorption. Of interest
    in this long distance simulation is that the resultant probe pulse is able
    to form its own steady-state soliton, as described in the study of matched
    pulses in chapter \ref{chp:nonlinear}. Rather than dissipating entirely, the
    probe pulse area $\theta_p$ (bottom, green) is held abruptly at around $z =
    $ \unit[$0.1$]{$L$} to a value of around $1 \pi$. The simultaneous
    propagating pulses first steepen toward the sech shape, but then broaden and
    slow due to the spontaneous decay. We see the large area of the \textsc{cw}
    probe decreases but doesn't disappear, and the combined pulse area $\theta =
    \sqrt{\theta_p^2 + \theta_c^2}$ (bottom, red dashed) finds its steady state at
    $2 \pi$. The pulses do not reach the end of the medium in the duration of the simulation, propagating a distance of $z = $ \unit[$0.7$]{$L$}.

    We may ask: what does it means to define a pulse area for an input
    \textsc{cw} field? For our purposes, we may take it to be arbitrarily large.
    Numerically, we integrate the Rabi frequency envelope over the duration of
    the simulation. The key point is that in the case that the combined pulse
    area is large enough to support simultaneous propagation, this arbitrarily
    large pulse area does not dissipate but is held.

    What happens for stronger pulses? In figures \ref{fig:pulse_cw_4pi_cmap} and
    \ref{fig:pulse_cw_6pi_cmap} we present results for larger-area pulses input
    on the same medium with the same \textsc{cw} probe field of $\Omega_p =
    \unit[10]{\Gamma}$. The coupling input pulses are again Gaussians of width
    $\tau_w = $ \unit[0.01]{$\tau_\Gamma$}.

    \begin{figure}%[h]
    \includegraphics[width=\linewidth]{figs/06_simultons/mb_vee_sit_plot_45pi_Ng1e4_fig1.pdf}
    \caption{
    Propagation of a Gaussian $4.5 \pi$ input coupling pulse with width
    \unit[$0.01$]{$\tau_\Gamma$} through a V-type medium addressed by a
    $\unit[10]{\Gamma}$ \textsc{cw} probe. (Top left) Propagation profile of the
    probe (green) and coupling (blue) fields. (Bottom left) Pulse areas of the
    fields and the total area (red dashed). (Right) Colourmaps of the real part
    of the complex Rabi frequencies $\Omega_{p}$ and $\Omega_{c}$.
    }
    \label{fig:pulse_cw_4pi_cmap}
    \end{figure}

    In figure \ref{fig:pulse_cw_4pi_cmap}, for the $4.5 \pi$ pulse, we see the
    coupling pulse break apart as we've seen previously. Again we see that the
    pulse kicks up a simultaneous pulse in the probe field as the absorption is
    initially reduced, which allows a small pulse area to move through, and this
    is carried on by the first resultant $2 \pi$ pulse.

    We may understand the reason that only one pulse propagates in the probe
    field by considering the evolution of the off-diagonal matrix elements we
    presented in figure \ref{fig:sim_0703_temp_210C_build0_coh}. For every two
    oscillations in $\rho_{02}$, the system evolves through one oscillation in
    $\rho_{01}$. This oscillation forms the probe component of a pulse that
    matches with the first coupling pulse and propagates as a simulton.

    \begin{figure}%[h]
    \includegraphics[width=\linewidth]{figs/06_simultons/mb_vee_sit_plot_65pi_Ng1e4_fig1.pdf}
    \caption{
    Propagation of a Gaussian $6.5 \pi$ input coupling pulse with width
    \unit[$0.01$]{$\tau_\Gamma$} through the V-type medium addressed by a
    $\unit[10]{\Gamma}$ \textsc{cw} probe. (Top left) Propagation profile of the
    probe (green) and coupling (blue) fields. (Bottom left) Pulse areas of the
    fields and the total area (red dashed). (Right) Colourmaps of the real part
    of the complex Rabi frequencies $\Omega_{p}$ and $\Omega_{c}$.
    }
    \label{fig:pulse_cw_6pi_cmap}
    \end{figure}

    In figure \ref{fig:pulse_cw_6pi_cmap}, for the $6.5 \pi$ pulse, we see that the coupling pulse breaks into three resultant pulses as we'd expect and the kicked up pulse area in the probe field is carried mostly by the first and third resultant $2 \pi$ pulses.

    [TODO: Clarify the argument.]

    These demonstrative simulations provide an interesting result: a portion of
    the \textsc{cw} probe field is in fact picked up by the strong pulse and
    carried along as a simultaneous pulse with the same width, at the same
    velocity, and capable of propagating over long distances.

  \subsection{Comparison with Experimental Data}

    The discovery that a strong coupling pulse has the effect of causing a long-
    distance propagating soliton in the \textsc{cw} probe leads us to consider
    the experimental results from the thin cell, using the parameters of section
    \ref{sec:simultons_experiment}, but imagining that the cell is longer.
    Recall that in the case of transitions on the the rubidium \textsc{d1} and
    \textsc{d2} lines we have distinct values for $g_{01}$ and $g_{02}$, as they
    are proportional to the square of the respective dipole moments, $d_{0j}^2$.
    This may affect the ability of the pulses to match and propagate.

    \begin{figure}[h]
      \includegraphics[width=\linewidth]
        {figs/06_simultons/mb_vee2g_15c_130p_0330t_230C_sb50_120vel010_10_050um_fig1.pdf}
      \caption{
      (Top) Comparison of numerical results (red) with experimental data (blue
      circles) for the normalised transmitted probe signal. The measured coupling
      pulse signal (grey filled area) has a width $\tau_w = $ \unit[$0.80$]{ns} $
      \equiv $ \unit[$0.029$]{$\Gamma_\tau$} in each case. The temperature is
      \unit[$230$]{\textdegree C}. (Middle) Colour map of the real part of the
      complex Rabi frequency for the probe field. The red dotted line marks $z =
      1L$. (Bottom) Colour map of the real part of the complex Rabi frequency for
      the coupling pulse.
      } 
      \label{fig:exp_result_single} 
    \end{figure}

    In figure \ref{fig:exp_result_single} we present again the comparison of
    numerical result and experimental data shown in figure
    \ref{fig:sim_data_0703_temp_210C}. In the experiment the temperature $T =
    \unit[230]{\textrm{\textdegree C}}$, and the peak pulse power is
    \unit[$85$]{mW}, and the same parameters are simulated. We again present the
    real part of $\Omega_p$ and $\Omega_c$ but continue the simulation over a
    longer distance, imagining that the cell is much longer at \unit[$50$]{$\mu
    m$} $ \equiv$ \unit[$25$]{$L$}. This will allow us to predict the long-
    distance behaviour.
