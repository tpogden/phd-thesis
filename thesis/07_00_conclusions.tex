\chapter[Conclusions]
  {Conclusions}
  \label{chp:conclusions}

  We began this thesis by describing in chapter \ref{chp:propagation} a model
  for the propagation of pulses in dense thermal atomic vapours, based on the
  Maxwell Bloch equations. We introduced the regimes of linear and nonlinear
  optics, and using the propagation model presented the concepts and simulated
  examples of two important nonlinear techniques: \textsc{sit} and the
  propagation of optical solitons and simultons in chapter \ref{chp:nonlinear},
  and three-level \textsc{eit} and the propagation, storage and retrieval of
  dark-state polaritons in chapter \ref{chp:polaritons}.

  In chapter \ref{chp:twophoton} we investigated the properties of two-photon
  excitation of a rubidium vapour with high-intensity beam, which required
  consideration of hyperfine pumping and thus the degenerate hyperfine structure
  of the coupled atomic states. We were able to rule out such two-photon
  excitation as a cause, and even as a significant contribution, of the dramatic
  increase in fluorescence observed in experimental data taken when scanning
  over resonance with the \textsc{d2} lines.

  Finally, in chapter \ref{chp:simultons}, we constructed a model to explain an
  interesting observed nonlinear effect, making use of the effects of
  \textsc{sit} and \textsc{eit} previously introduced, indicating that this is,
  to our knowledge, a first observation of simulton propagation in an atomic
  vapour. Moreover, by considering the behaviour of the system over longer
  propagation distances, we described an approach for facilitating propagation
  of weak pulses through dense atomic vapours.

  The \textit{OpticalBloch} software package developed for solving the Maxwell
  Bloch equations is presented in appendix \ref{apx:mb_eqns} and was used for
  producing the numerical results throughout this thesis. It has been designed
  to be easily extensible to other systems, including those with many atomic
  levels and with multiple co-propagating fields.

  The simulated results of simulton propagation presented in chapter
  \ref{chp:simultons} give a good quantitative fit to the experimental data over
  a range of powers and temperatures, including for the arrival time and
  relative heights of the first peak in the probe signal. This gives us
  confidence in the model but those fits (see figures
  \ref{fig:exp_result_temp_dep} and \ref{fig:exp_result_power_dep}) are less
  than perfect, notably in the oscillations subsequent to the initial peak.
  Improving this fit should be a goal of future work. Now that we have an
  understanding of the propagation mechanism, experiments can be designed to
  monitor all of those parameters we now know to be important, including the
  coupling pulse profile before and after transmission through the vapour cell.
  We can also use our understanding of the pulse breakup Rabi frequencies to
  carefully measure the coupling strength involved the atom-light interaction.
  
  Improvements should also be considered for the theoretical model. We can
  adjust the simulated coupling pulse from a pure Gaussian to match the exact
  profile input on the medium. We justified averaging over the hyperfine
  sublevels in excited states, however an obvious and useful extension to the
  model will be to include the angular momentum structure developed in chapter
  \ref{chp:twophoton} to account for any effects due to degeneracy and optical
  pumping.

  Experimental work has begun to investigate propagation of simultons over
  longer distances in micron-scale Caesium vapour cells, which will enable us to
  test the feasibility of weak propagation using the simulton scheme. 

  An exciting extension of the scheme would be in the possibility of coupling
  using Rydberg states in order to introduce strong dipole-dipole interactions
  between atoms to mediate interactions between solitons, with applications for
  photonic quantum information
  processing.\cite{Maxwell2013,Maghrebi2015,Peyronel2012}
