\section{Numerical Results}
  \label{sec:twophoton_results}

  \subsection{Weak Beam Spectra}

    Before we investigate the effect of a strong field on the atomic medium, it
    is useful to look at the results of an applied field in the weak probe
    regime. As discussed in chapter \ref{chp:propagation}, in the weak field it
    is possible to derive analytic expressions for the spetral Lorentzian (or
    Voigt, when broadened)  lineshape corresponding to the imaginary part of the
    coherences and thus for the expected absorption profiles.

    The ElecSus software package\cite{Zentile2015} calculates transmission and
    susceptibility spectra for weak probes in thermal alkali metal vapours, in
    excellent agreement with experimental data.\cite{Siddons2008,Weller2011} We
    use this tool as a reference with which to compare the results for our
    model at weak field, before we take our model beyond the constraints of the
    weak probe regime.

    \begin{figure}[]
    \includegraphics[width=\linewidth]
        {figs/05_twophoton/rb87_d2_hf_solve_scan_g2a_fig3.pdf}
    \includegraphics[width=\linewidth]
        {figs/05_twophoton/rb87_d2_hf_solve_scan_h1a_fig3.pdf}
    \caption{
    Simulated transmission (red) of a weak ($I~=~\unit[10^{-6}]{W~cm^{-2}}$)
    probe beam scanned across the  $F = 1 \rightarrow F$ \textsc{d2} lines in a
    \unit[1]{cm}long vapour cell of rubidium 87 is compared with the result of
    the ElecSus program (grey dashed). Doppler broadening is neglected and the
    number density $N = \unit[7.5\cdot10^{15}]{m^{-3}}$. The simulated transit
    time \unit[2]{$\mu$s}.
    }
    \label{fig:weak_d2} 
    \end{figure}

    In figure \ref{fig:weak_d2} we show the results of a simulated scan of a
    weak probe (with intensity $I = $\unit[$10^{-6}$]{W cm$^{-2}$}) with no
    Doppler broadening, corresponding to a temperature close to \unit[$0$]{K},
    but with a number density of $N = \unit[7.5\cdot10^{15}]{m^{-3}}$,
    corresponding via  vapour pressure equations\cite{Zentile2015} to a
    temperature $T = \unit[20]{\text{\textdegree C}}$. The removal of broadening
    is an artificial constraint intended to separate the Lorentzian lineshape of
    the coherence terms (and thus absorption profile) without needing to include
    the Gaussian convolution. The simulated length of the medium in this case is
    \unit[1]{cm}.

    To obtain spectra for closed systems it is typical to compute the density
    matrix elements in the steady state, \ie by setting  $\partial \rho /
    \partial t = 0$ in the Lindblad master equation (\ref{eqn:lindblad}). This
    requires less computation than integrating the differential equations over
    time, but is not possible in this system because of hyperfine pumping. As $t
    \rightarrow \infty$, all of the population is pumped to the other ground
    state. Instead, we solve the master equation up to an average transit time
    of atoms in the beam, in this case \unit[2]{$\mu$s}, and present the
    resulting density matrix elements at this point in the time evolution.

    % As we wish to , we 

    The top subplot covers the transition from ground state hyperfine levels $F
    = 1$ to excited state hyperfine levels $F' = \{ 0, 1, 2 \}$ and the bottom
    subplot covers the transition from hyperfine levels $F = 2$ to excited state
    hyperfine levels $F' = \{ 1, 2, 3 \}$. We see good agreement between the
    optical Bloch model and the ElecSus result for the positions, widths and
    amplitudes of the absorption troughs.

  \subsection{Strong Beam Spectra}

    Next we move beyond the weak-probe approximation to investigate the response
    of the atom is to a strong beam, by way of the populations of the hyperfine
    levels
    \begin{equation}
      \rho_{FF} = \sum_{m_F} \Tr \left[ \Ket{F m_F} \Bra{F m_F} \right].
    \end{equation}

    \begin{figure}[]
    \includegraphics[width=\linewidth]{figs/05_twophoton/rb87_d2_hf_solve_scan_h2a_fig1.pdf}
    \caption{
    Populations of the $5^2\rm{S}_{\nicefrac{1}{2}}$ ground states (top) and
    $5^2\rm{P}_{\nicefrac{3}{2}}$ excited states (bottom)  after interaction
    with a strong beam  scanned across the  $F = 1 \rightarrow F'$ \textsc{d2}
    lines with a transit time of \unit[2]{$\mu$s}.
    }
    \label{fig:strong_d2_f2} 
    \end{figure}

    \begin{figure}[]
    \includegraphics[width=\linewidth]{figs/05_twophoton/rb87_d2_hf_solve_scan_g3a_fig1.pdf}
    \caption{
    Populations of the $5^2\rm{S}_{\nicefrac{1}{2}}$ ground states (top) and
    $5^2\rm{P}_{\nicefrac{3}{2}}$ excited states (bottom)  after interaction
    with a strong beam scanned across the  $F = 2 \rightarrow F'$ \textsc{d2}
    lines with a transit time of \unit[2]{$\mu$s}.
    }
    \label{fig:strong_d2_f1} 
    \end{figure}

    Again we'll consider just the $5^2\rm{S}_{\nicefrac{1}{2}}$ ground state
    manifold and the $5^2\rm{P}_{\nicefrac{3}{2}}$ excited state manifold
    representing the \textsc{d2} transition. In figures \ref{fig:strong_d2_f1}
    and \ref{fig:strong_d2_f2} we show the results of simulated scan, again over
    the \textsc{d2} lines of rubidium 87, but this time with a strong beam of
    intensity $I = $ \unit[$1$]{W cm$^{-2}$}. This is beyond the saturation
    intensity and in the regime where power broadening has a significant effect
    on the transmission profile. The transit time is again \unit[2]{$\mu$s}.

    In figure \ref{fig:strong_d2_f1} we show the populations of the hyperfine
    $5^2\rm{S}_{\nicefrac{1}{2}}$ ground states and
    $5^2\rm{P}_{\nicefrac{3}{2}}$ excited states as the probe is scanned across
    the $F = 1$ to $F' = \{ 0, 1, 2 \}$ transitions. We see that when we are far
    off-resonance the population remains in the initial state, evenly divided
    between the two ground state hyperfine levels. As the scan crosses the
    resonance lines we see population is removed from the $F = 1$ state and
    populates the excited hyperfine states according to that transition's
    relative transition strength. The lineshapes are now much broader than the
    natural linewidth due to power broadening\cite{loudon2000quantum}. The $F' =
    2$ population has a double- peaked lineshape. This is due to hyperfine
    pumping saturating this transition at this probe strength, \ie on resonance
    from $F = 1 \rightarrow F' = 2$ all of the population decays to the $F = 2$
    state within \unit[2]{$\mu$s} at this intensity such that the population in
    the $F' = 2$ state is limited.

    In figure \ref{fig:strong_d2_f2} we show the same populations as in figure
    \ref{fig:strong_d2_f1}, but for the probe scanned across the $F = 2$ to $F'
    = \{ 1, 2, 3 \}$ transitions. We again see that far off-resonance the
    populations remain in their initial condition state, evenly split between
    the two ground state hyperfine levels. As the scan crosses the resonance
    lines, this time we see conversely that the population is removed from the
    $F = 2$ state to populate the excited hyperfine states according to the
    transition strengths. This time it is the $F' = 1$ and $F' = 2$ transitions
    that are saturated by the hyperfine pumping such that the initial state
    population is limited on resonance after \unit[2]{$\mu$s} at this intensity.

    \begin{figure}[]
    \includegraphics[width=\linewidth]{figs/20_135_placeholder.pdf}
    \caption{
    Excited state populations
    }
    \label{fig:strong_Dstate_pop} 
    \end{figure}    

    [TODO describe 5D state pop plot]

  \subsection{Fluorescence with High-Intensity Beam}

    Now that we have investigated population of the
    $5^2\rm{S}_{\nicefrac{1}{2}}$ hyperfine ground states as the probe is
    scanned across the \textsc{d2} lines, both for a weak probe and a strong
    probe, we will move onto adding in the two-photon excited states ---
    firstly, the $5\rm{D}_{\nicefrac{3}{2}}$ and $5\rm{D}_{\nicefrac{5}{2}}$
    states coupled near resonance, and the $6\rm{P}_{\nicefrac{1}{2}}$ and
    $6\rm{P}_{\nicefrac{3}{2}}$ decay channels. Decay form the
    $6\rm{P}_{\nicefrac{1}{2}}$ state is the source of optical fluorescence at
    \unit[422]{nm}. We wish to observe if there is \textit{enough} population in
    these exited states through single-atom processes to account for the
    fluorescence shown in figure \ref{fig:blue_flourescence}.

    \begin{figure}[]
    \includegraphics[width=\linewidth]
        {figs/05_twophoton/rb87_5spd6p_hf_solve_scan_f01_fig2.pdf}
    \caption{
    Normalised simulated population of the $6\rm{P}_{\nicefrac{1}{2}}$ state for
    light input across a GHz detuning range covering the \textsc{d2} lines, and
    over a range of number densities $N$ (and thus temperatures $T$).
    [TODO What is the peak?]
    }
    \label{fig:blue_plot_model} 
    \end{figure}

    In figure \ref{fig:blue_plot_model} we show the normalised population of the
    $6^2\rm{P}_{\nicefrac{1}{2}}$ state as a very strong probe, with $I =
    \unit[100]{W/cm^2}$, is scanned across the full range of the \textsc{d2}
    lines. At this intensity we see that the power broadening is such that the
    hyperfine structure of the excited states is not visible and we see just two
    peaks representing transition from the $F = 1$ and $F = 2$ ground states.

    The peak population is only on the order of $10^{-11}$, such that we expect
    one in a hundred billion atoms to be excited.