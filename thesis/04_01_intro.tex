\section{Introduction}
  \label{sec:polaritons_intro}

  In chapter [ref] we introduced the nonlinear optical phenomenon known as
  \textsc{sit}, which allows a particular pulse profile, an optical soliton, to
  travel unimpeded through a medium that it would ordinarily find to be opaque.
  In this chapter we introduce another familiar technique for overcoming the
  absorptive effect of a medium, known as \textit{electromagnetically induced
  transparency} (\textsc{eit}).[REFS]

  The phenomenon of \textsc{eit} is unlike \textsc{sit} in that it does not
  specify a particular pulse area as having the ability to be transmitted,
  though there are bandwidth limitations as we will discuss. Instead,
  \textsc{eit} makes use of a probe transition being strongly coupled to a
  second transition in a three-level medium to drive coherence in atomic states
  and set up destructive interference between excitation channels. This results
  in a useful large dispersive nonlinear susceptibility around the probe
  resonance.

  It is possible to construct such systems in a $\Xi$-type three-level medium
  [ref] but we will focus on the most-commonly used $\Lambda$-type system, which
  is the most convenient due to the possibility of using two lower levels with
  negligible decay, which allows for a metastable dark state.

  In addition to the result of transparency, if we are able to manipulate
  parameters such as the coupling field power and the atomic density, we gain
  powerful control over the propagation of light in the medium. This includes
  the ability to adjust the speed of a propagating pulse and its spatial extent.
  We will present the quasiparticle known as the \textit{dark-state polariton},
  useful for understanding such \textsc{eit} propagation.

  Of particular interest is the possibility of storing and retrieving a pulse of
  light in an optically dense \textsc{eit} medium with a strong, time-dependent
  coupling field, which provides a mechanism for the implementation of quantum
  memory[refs], a key requirement for quantum information processing
  (\textsc{qip}).[ref]


